%!TEX ROOT=ctutest.tex

\chapter{Komunikační řetězec – IoT aplikace}
\question{Popis komunikačního řetězce v teoretickém úvodu?}
\answer{V BP může být v teor. části obecnější popis = bez konkrétních součástek, nastavení apod., v části Realizace pak navázat popisem konkrétní implementace - součástky, nastavení, atd. }


\section{LoRa Node}
    LoRa Node je koncové zařízení, které komunikuje se senzory, zpracovává naměřená data a odesílá je v podobě LoRa paketů do LoRa brány.

\subsection{Výběr komponent}

\subsubsection{Mikrokontrolér (MCU) – STM32L0RZ73 }

    Výběr výpočetní jednotky byl ovlivněn zejména dvěma požadavky – dostatečným výpočetním výkonem, který je třeba pro zpracování signálu z akcelerometru, a nízkou spotřebou. Tyto požadavky jdou ovšem zcela proti sobě, a nakonec byl tak jako kompromis použit procesor z „low power“ série mikrokontrolérů od firmy STM – STM32L0RZ73. 
    
    \question{Doplnit o specifikaci MCU??}
    
\answer{Ano, v BP udělat přehled relevantních parametrů a architektury}

\subsubsection{LoRa modul – I-NUCLEO-SX1272D}
    $\text{LoRa}^{\text{TM}}$ je uzavřená technologie, kterou licencuje a vyvíjí firma SemTech. Všechny čipy podporující tuto modulaci nesou označení SX127x. Na trhu se nachází nepřeberné množství LoRa modulů, které se liší především výstupním výkonem, komunikačním rozhraním a cenou.
    
    Protože jsem chtěl vytvořit vlastní LoRa bránu a nevyužívat oficiální brány, bylo třeba použít pouze LoRa modul a nikoliv LoRaWAN modul, který má v sobě implementovaný LoRaWAN protokol a slouží právě pro komunikaci s oficiálními branami. K tomuto účlu posloužil LoRa modul s čipem SX1272 přímo od společnosti ST, který byl součástí vývojářského kitu, který jsem dostal a který lze zapojit přímo do Nuclea.
    
    
    %\href{https://www.st.com/en/evaluation-tools/i-nucleo-sx1272d.html}{Source}




\subsubsection{Teplotní senzor – TMP75}

    TMP75 je digitální teplotní čidlo od firmy Texas Instruments, které pro měření využívá 12bitový převodník. Jedná se o poměrně rozšířený senzor komunikující po I2C sběrnici, který je díky své malé spotřebě vhodný zejména do zařízení, které jsou poháněny baterií.
    
    \question{Doplnit o specifikaci senzorů??}

\subsubsection{Senzor vibrací – ADXL1001}

    \todo{!!!Podle čeho vybrat akcelerometr do naší aplikace???}
    
    \answer{rozsah, noise floor u analogového/ rozlišení u digitálního, šířka pásma/vzorkovací frekvence}




\subsection{Návrh zapojení}

    Zapojení LoRa Node je vyřešeno za pomocí vývojářského kitu NUCLEO a nepájivého pole, do kterého jsou připojeny senzory. Jedná se zatím pouze o první verzi pro testovací účely. Oba dva senzory jsou napájeny z Nuclea a fungují na 3.3 V logice. K senzorům jsou přidány kondenzátory pro blokování napájení. Výsledné zapojení lze vidět na obrázku...
    
    \todo{!!!Doplnit Schéma zapojení}
    
    \todo{!!!Doplnit foto}


\subsection{Softwarové nástroje}

    Softwarový vývoj postavený na platformě ST může probíhat několika odlišnými způsoby. Lze využívat knihovnu obsahující pouze registry procesoru – CMSIS, nízkoúrovňovou knihovnu LL (Low Level) nebo již komplexní knihovny HAL (Hardware Abstraction Layer). V této práci byla použita kombinace všech tří přístupů. Obsluha komunikace MCU přes UART byla například naprogramována pouze pomocí CMSIS, obsluha AD převodníku pomocí LL a obsluha I2C pomocí HAL.
    
    \question{Mělo by to byt všechno unifikované???}
    
    \answer{Nemusí být, pokud je to zdůvodnitelné - např. někde něco chybí}

\subsection{Funkčnost LoRa Node}

    Průběh programu je následující. Po restartování MCU dojde k inicializaci periferií a příslušné nastavení GPIO pinů. V dalším kroku dochází ke čtení hodnot ze senzorů, jejich naformátování a vytvoření zprávy, která je odeslána prostřednictvím LoRa modulu. 
    Po odeslání paketu přechází zařízení do SLEEP módu, ze kterého je po 30 sekundách pomocí RTC obvodu probuzeno a dochází opět k dalšímu čtení dat ze senzorů a odeslání zprávy.

\subsection{Způsob komunikace – formát odeslaných dat}

    \todo{!!!Vymyslet přesně daný formát zprávy}




\section{LoRa brána}

    \todo{Otestovat dosah brány}
    
    \todo{Změřit spotřebu a propustnost}
    

    Brána (Gateway) slouží jako prostředník mezi jednotlivými koncovými zařízeními (LoRa Nodes) a cloudovou službou. Její hlavní funkcí je tedy příjem LoRa paketů a jejich přeposílání přes IP protokol do cloudu.

\subsection{Výběr komponent}
    
    \subsubsection{Mikrokontrolér – Raspberry Pi}
        Vlastní LoRa brána může být postavena spíše minimalisticky za použití určitého mikrokontroléru, tedy se slabým výkonem, ale nízkou cenou a spotřebou – například jako další STM32 jednotka s ethernet nebo WiFi modulem nebo pomocí oblíbeného vývojářského čipu ESP8266/ESP32. 
        Opačný přístup by bylo postavení brány s výpočetně silnějším počítačem a operačním systémem, tedy s velkým výkonem, ale vyšší cenou a spotřebou.

        Pro realizaci brány bylo nakonec použité \textbf{Raspberry Pi}, konkrétně verze 3 model B s operačním systémem Raspbian. 
        Motivací výběru Raspberry po zvážení kritérií popsaných výše byl zejména dostatečný výpočetní výkon, vývojářská podpora a výhody operačního systému. Spotřeba brány není tak důležité kritérium jako u koncových zařízení, protože brána může mít stálý zdroj napájení. Raspberry byl tedy jakýsi kompromis mezi výkonem a cennou. Další plus, které toto řešení nabízí je veliká univerzálnost – na RPi lze naprogramovat mnoho aplikací, které by přizpůsobily chování celého systému podle konkrétních požadavků zákazníka (například lokální záloha dat z LoRa Nodes, pouze lokální řešení bez cloudové služby \ldots).
        
        \question{Doplnit o technické parametry Raspberry?}
        
        \answer{pouze ty podstatné}
        
    \subsubsection{LoRa modul – I-NUCLEO-SX1272D}

        Pro komunikaci s koncovými zařízeními byl využit stejný LoRa modul jako pro koncová zařízení – \textbf{I-NUCLEO-SX1272D}.

        Bohužel tento modul je určený pro vývojovou desku Nucleo, čemuž odpovídá i jiná DPS, a proto bylo třeba propojit GPIO piny Raspberry a modulu drátky. V další verzi by bylo určitě vhodné vytvořit desku s LoRa modulem, která by přímo seděla na rozměry Raspberry a nemusela se tak propojovat.
        
    \subsubsection{LCD displej – LCD 1602A}
        I když lze stav všech koncových LoRa Nodes lze kontrolovat přes síť The Things Network, je vhodné použít nějaký „offline“ způsob indikace stavu celého systému, a proto byl k Raspberry připojen LED displej s posuvným registrem, který zobrazuje informace o připojených LoRa Nodes. Samotnému výběru displeje jsem nevěnoval příliš pozornost, protože se jednalo pouze o dodatečnou funkcionalitu.
        
\subsection{Návrh zapojení}

        Výsledné zařízení lze vidět na obrázku...
    
        \todo{Vytvořit obrázek zapojení}
        
\subsection{Softwarové nástroje}

    Proti řešení brány pouze s mikrokontrolérem je u malého linuxového počítače velkou výhodou možnost využití skriptovacího programovacího jazyka Python, který díky mnoha knihovnám usnadňuje práci vývojáři a který byl také použit pro naprogramování brány.

    Pro obsluhu GPIO pinů a periferií Raspberry jako SPI a UART byla využita knihovna \textit{wiringpi}. Samotná komunikace Raspberry a LoRa modulu probíhala přes SPI komunikaci zápisem a čtením jednotlivých registrů čipu. 
    Pro komunikaci s cloudovou službou The Things Network byla využita knihovna \textit{socket}, který umožňuje přístup k linuxovému rozhraní BSD soket.



\subsection{Způsob komunikace}
    Po počáteční inicializaci brána čeká v módu SLEEP na přijetí LoRa paketu. Přijetí paketu je doprovázeno zablikáním indikační LED diody a aktualizací údajů na displeji. Dále brána dekóduje zprávu a zkontroluje CRC. 
    
    \question{Doplnit o popis CRC???}
    
    \textit{CRC (Cyclic Redundancy Check) je algoritmus...}
    
    \answer{stačilo by jen napsat, který polynom používáte}

    V případě, že CRC nesouhlasí je daný paket zahozen, nebyly implementovány žádné ARQ algoritmy pro  opakované odeslání chybných dat. Pokud paket dojde v pořádku je jeho obsah (naměřené hodnoty) překopírován do textového řetězce ve formátu JSON  odeslán jako TCP soket na veřejnou IP adresu jednoho ze servrů The Things Network.
    
     \question{Doplnit o popis JSON???}
     
     \answer{ano, ale jen relevatní informace}

    \textit{JSON udává způsob zápisu dat, který je nezávislý na platformě a je v textové podobě. Struktura JSON je definována  množinou pravidel...}
    
    \question{Doplnit o obrázek algoritmu???}

\answer{finální flowchart ano, asi stačí až po finálním rozběhnutí}


  
