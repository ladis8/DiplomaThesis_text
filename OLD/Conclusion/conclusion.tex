%!TEX ROOT=ctutest.tex
\chapter{Závěr}
    Tato práce popisuje vývoj komplexního systému pro monitorování stavu rotačních průmyslových zařízení pomocí analýzy vibrací a teploty.
    Pro sběr dat z akcelerometru a teplotního čidla, jejich předzpracování a výpočtu signálových analýz byla navržena měřicí jednotka postavená na mikrokontroléru STM32L072 a komunikující s centrální jednotku přes LoRa síť prostřednictvím modulu RFM95W. Mezi nejdůležitější signálové vlastnosti, které jednotka vypočítává, patří hlavně RMS, krest faktor, kurtosis ratio a spektrální amplitudová maxima. Tato data slouží k odhadu možné poruchy stroje, jak bylo ověřeno během experimentu na reálném motoru. Měřicí jednotky jsou uspořádány ve hvězdicovité topologii, přihlašují se, žádají o zaslání konfigurace a odesílají naměřená a vypočtená data centrální jednotce.
    
    Ve středu systému stojí centrální jednotka (LoRa brána), prostředník mezi měřicími jednotkami a aplikačním serverem, jež řídí rádiovou komunikaci a je postavena na mikropočítači Raspberry Pi využívající stejný LoRa modul jako měřicí jednotky. Díky tomuto modulu je brána sice mnohem levnější, ale pouze jednokanálová, dovolující komunikaci v jednu chvilku pouze s jednou jednotkou. Proto byl brán zvláštní ohled na efektivitu a rychlost rádiové komunikace.
    
    Aplikační server využívá platformy Node-RED a databázi PostgreSQL a je od centrální jednotky zcela logicky oddělen (v rámci vytvořené demo aplikace ale také běží na Raspberry Pi). Centrální jednotka odesílá naměřená data do aplikačního serveru přes UDP soket a server je ukládá do databáze. Přes webové rozhraní poté může uživatel analyzovat data prostřednictvím mnoha grafů a zároveň pomocí něj konfigurovat jednotlivé měřicí jednotky.
    
    Během celé práce byl kladen důraz především na tuto konfigurovatelnost. Uživatel si může vytvořit pro každou jednotku vlastní konfiguraci, která je poté centrální jednotkou nahrána do příslušného zařízení. Pozornost byla také věnována dodržení moderních konceptů IIoT a cloud computingu, a proto jsou naměřená data a konfigurační údaje uložena v databázi a přístupná aplikacím třetích stran přes RESTový server k dalším analýzám.
    
    Na závěr byl celý systém otestován na reálném simulátoru vibrací, při kterém bylo rotační zařízení vždy monitorováno nejdříve bez poruchy a poté s určitou poruchou. Následně byly monitorované veličiny porovnány a byl zhodnocen jejich přínos a vypovídající hodnota.
    
\section{Budoucí práce}
    Zkouška funkcionality celého systému na reálném motoru, která byla bohužel provedena až v posledních týdnech, pomohla poodhalit náročnost a komplexnost celé vibrační diagnostiky jako takové. Aplikované analýzy signálu se mohou rychle stát bezcennými, pokud nejsou použity pro správné frekvenční rozsahy, a jejich vypovídající hodnota je tedy maximálně závislá na preprocessingu původního signálu (odstranění střední hodnoty a základní filtrace nestačí). Kdybychom tedy chtěli naplno využít potenciál například krestfaktoru a kurtosis ratio, bylo by nutné před jejich použitím využít pásmové propusti a signál vyfiltrovat pro požadované frekvence. Pro detailnější a univerzálnější diagnostiku by také bylo vhodné použít dalších složitějších signálových analýz jako například kurtogramu, jež byly ale pro implementaci na STM32L0 příliš výpočetně náročné, nebo využít zcela odlišného senzoru a analýz jako HFRT (High Frequency Resonance Technique). V dalších verzích by se tedy také mohl objevit výpočetně mocnější mikrokontrolér z nových řad STM.
        
    Co se komunikačního řetězce týče, tak pro nasazení systému i s větším počtem monitorovacích jednotek by bylo vhodné postavit bránu na dražších LoRa čipech určených přímo pro multikanálové brány, čímž by se docílilo rychlejší a spolehlivější komunikace. LoRa jako bezdrátová technologie se poměrně ověřila, ale na druhou stranu by bylo třeba provést testy dosahu, propustnosti v průmyslovém prostředí a také testy výdrže baterie. Během práce byla komunikace testována pouze v „kancelářském“ prostředí do vzdáleností desítek metrů.\\
    Teplotní čidlo TMP75 není příliš vhodné do průmyslového prostředí a mělo by se nahradit vhodnějším senzorem.
    
    Z hlediska softwaru se Node-RED ukázal jako dobrý nástroj pro jednodušší aplikace. Jakmile se ale jeho kód postupně rozrůstal, bylo patrné, že pro komplexnější aplikace je flowchartové programování značně limitující a nepřehledné, a dále by proto bylo výhodnější využít jeden z moderních webových serverů jako Django nebo Node.js.     
        